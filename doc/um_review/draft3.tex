% To build, please type: pdflatex draft1.tex

\documentclass[12pt,twoside]{article}
\usepackage[bf,small]{caption}
\usepackage[letterpaper,hmargin=1in,vmargin=1in]{geometry}
\usepackage{paralist} % comapctitem, compactdesc, compactenum
\usepackage{titlesec}
\usepackage{titletoc}
\usepackage{times}
\usepackage{hyperref}
\usepackage{algorithmic}
\usepackage{graphicx}
\graphicspath{{./graphics/}}
\usepackage{xspace}
\usepackage{verbatim}
\usepackage{url}
\usepackage{float}
\hyphenation{Sub-Bytes Shift-Rows Mix-Col-umns Add-Round-Key}

\setlength{\parskip}{12pt}
\setlength{\parindent}{0pt}

\newcommand{\hdb}{\emph{hashdb}\xspace}
\newcommand{\libhdb}{\emph{libhashdb}\xspace}
\newcommand{\bulk}{\emph{bulk\_extractor}\xspace}
\newcommand{\hashid}{\emph{hashid}\xspace}
\newcommand{\hid}{\emph{hashid}\xspace}
\newcommand{\mdd}{\emph{md5deep}\xspace}
\newcommand{\bev}{\emph{Bulk Extractor Viewer}\xspace}
\newcommand{\fiwalk}{\emph{fiwalk}\xspace}

\begin{document}
Draft review 3.

I like the use of filename extensions.

\texttt{mock\_video\_db.hdb} has redundancy.  Plese use
\texttt{mock\_video.hdb}.
Lets use the \texttt{.xml} filename extension for all our DFXML files.
Instead of \texttt{mock\_video\_dfxml}, please use \\
\texttt{mock\_video\_hashes.xml}.
\texttt{intersection.hdb} instead of \texttt{intersection\_hashdb.hdb}.

Lets add \texttt{.hdb} to the \hdb database name that \bulk creates
when it is in \texttt{import} mode.
If \texttt{hashdb.hdb} seems okay, lets use that.
I'll change the \bulk \hid scanner to comply.

I redid the demos at
\url{https://github.com/simsong/hashdb/wiki}.
They should now reflect this naming convention.

From my discussion with Michael, who met with our sponsor:
\begin{compactitem}
\item They couldn't get the tools to work
because they had an old Command window open.
Please add: If you had a Command window open,
please close and reopen it so that Windows can find the \hdb tool.
\item They thought \texttt{hashes\_not\_inserted\_wrong\_hash\_block\_size}
indicated an error, but it is expected
since the remaining bytes of every file is likely to not be the exact
hash block size.
For now, please add this fact so they understand these variables better.
I don't know if I'll be able to fix this because it would lose information.
\end{compactitem}

It seems figures and listings tend to appear before they are referenced.
If you can, please format them to appear soon after.  Thanks.

Section 2.2: They don't need to know the exact length.
Please drop sentence "Note that the length... and 32, respectively".

Section 2.3: settings.xml: "settings the user requested
when the 'block hash database' was created", not when the \hdb was created.

Section 3.2.3: "In this case, each hash values shown...".
I don't get it.
Duplicate hash values get stored when the hash value is the same
and any of the source information is different.
A hash value does not have a duplicate if there is only one set of source
information citing that hash value.

Section 3.2.3: "benign" seems confusing.
How about "such as 'brand new' operating system images".

Table 4: \texttt{expand\_identified\_blocks}:
Usage and description are confusing.
Lets use usage:
\begin{verbatim}
expand_identified_blocks <hashdb> <identified_blocks.txt>
\end{verbatim}

For description please append text:
"for each hash feature in the \texttt{identified\_blocks.txt} input file."
to indicate that source information is printed for hashes in
\texttt{identified\_blocks.txt}, and not for each hash in the \hdb.

Section 4.4: You still say "Bloom filters can be used to quickly indicate
whether or not a hash value is found in the database".
This implies Bloom filters quickly indicate when a hash is in the database.
This is not true.  Please fix it.

Section 6: Please use the \texttt{bulk\_extractor-users} Google group instead.

\end{document}

